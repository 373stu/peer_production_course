\documentclass[]{article}
\usepackage{lmodern}
\usepackage{amssymb,amsmath}
\usepackage{ifxetex,ifluatex}
\usepackage{fixltx2e} % provides \textsubscript
\ifnum 0\ifxetex 1\fi\ifluatex 1\fi=0 % if pdftex
  \usepackage[T1]{fontenc}
  \usepackage[utf8]{inputenc}
\else % if luatex or xelatex
  \ifxetex
    \usepackage{mathspec}
  \else
    \usepackage{fontspec}
  \fi
  \defaultfontfeatures{Ligatures=TeX,Scale=MatchLowercase}
\fi
% use upquote if available, for straight quotes in verbatim environments
\IfFileExists{upquote.sty}{\usepackage{upquote}}{}
% use microtype if available
\IfFileExists{microtype.sty}{%
\usepackage{microtype}
\UseMicrotypeSet[protrusion]{basicmath} % disable protrusion for tt fonts
}{}
\usepackage[margin=1in]{geometry}
\usepackage{hyperref}
\hypersetup{unicode=true,
            pdftitle={Wikidata},
            pdfauthor={Brenna Wheeler},
            pdfborder={0 0 0},
            breaklinks=true}
\urlstyle{same}  % don't use monospace font for urls
\usepackage{graphicx,grffile}
\makeatletter
\def\maxwidth{\ifdim\Gin@nat@width>\linewidth\linewidth\else\Gin@nat@width\fi}
\def\maxheight{\ifdim\Gin@nat@height>\textheight\textheight\else\Gin@nat@height\fi}
\makeatother
% Scale images if necessary, so that they will not overflow the page
% margins by default, and it is still possible to overwrite the defaults
% using explicit options in \includegraphics[width, height, ...]{}
\setkeys{Gin}{width=\maxwidth,height=\maxheight,keepaspectratio}
\IfFileExists{parskip.sty}{%
\usepackage{parskip}
}{% else
\setlength{\parindent}{0pt}
\setlength{\parskip}{6pt plus 2pt minus 1pt}
}
\setlength{\emergencystretch}{3em}  % prevent overfull lines
\providecommand{\tightlist}{%
  \setlength{\itemsep}{0pt}\setlength{\parskip}{0pt}}
\setcounter{secnumdepth}{0}
% Redefines (sub)paragraphs to behave more like sections
\ifx\paragraph\undefined\else
\let\oldparagraph\paragraph
\renewcommand{\paragraph}[1]{\oldparagraph{#1}\mbox{}}
\fi
\ifx\subparagraph\undefined\else
\let\oldsubparagraph\subparagraph
\renewcommand{\subparagraph}[1]{\oldsubparagraph{#1}\mbox{}}
\fi

%%% Use protect on footnotes to avoid problems with footnotes in titles
\let\rmarkdownfootnote\footnote%
\def\footnote{\protect\rmarkdownfootnote}

%%% Change title format to be more compact
\usepackage{titling}

% Create subtitle command for use in maketitle
\providecommand{\subtitle}[1]{
  \posttitle{
    \begin{center}\large#1\end{center}
    }
}

\setlength{\droptitle}{-2em}

  \title{Wikidata}
    \pretitle{\vspace{\droptitle}\centering\huge}
  \posttitle{\par}
    \author{Brenna Wheeler}
    \preauthor{\centering\large\emph}
  \postauthor{\par}
      \predate{\centering\large\emph}
  \postdate{\par}
    \date{10/15/2019}


\begin{document}
\maketitle

\hypertarget{introduction}{%
\section{Introduction}\label{introduction}}

According to its own Wikidata entry, Wikidata began in 2012 as a ``free
knowledge database project hosted by Wikimedia and edited by
volunteers'' (``Wikidata'' 2019). As with many projects under the
Wikimedia Foundation auspices, Wikidata relies heavily upon these
volunteers for a wide variety of tasks, including generating content,
repairing vandalism, and organizing the entire Wikidata volunteer
community. The fulfillment of these tasks by volunteers reveals a
successful adaptation of Peer Production methods from the Free and Open
Source Software community, specifically Commons-Based Peer Production
(CBPP). For this paper, I will be examining Wikidata's adaptation of
Peer Production using the three characteristics of Peer Production
established by Benkler (2016):

\begin{itemize}
\item
  ``Harnessing diverse motivations''
\item
  ``Decentralization of conception and execution''
\item
  ``Separation of governance and management from property and contract''
\end{itemize}

The adaptation of Peer Production has created a large database of
structured linked data, which anyone can access and use. Finally, I will
be looking at Wikidata's relationship with the famous Wikipedia, and how
the two Peer Production platforms inform one another.

\hypertarget{harnessing-diverse-motivations}{%
\section{Harnessing Diverse
Motivations}\label{harnessing-diverse-motivations}}

Wikidata harnesses a diverse set of motivations by allowing anyone to
easily contribute to the project. As soon as a user is introduced to the
homepage, they are immediately provided information about how to
contribute under the ``Get Involved'' section, which links to additional
information about Wikidata, tutorials for adding data, a guide to using
data, and the community portal.

To get users familiar with creating and editing entries, Wikidata has
created two tutorials: ``Items'' and ``Statements''. These tutorials
walk a user step-by-step through the process of adding data by allowing
the user to edit the page for ``Earth'' in a sandbox environment. Each
step has additional details about rules and best practice for properly
logging data. By completing these tutorials, users can learn how to
contribute information that is consistent and properly linked to the
data currently available.

Once a user has completed the tutorials, Wikidata provides a series of
WikiProjects to get new users involved in areas of the database in need
of the most help. The first project linked on the Wikidata Introduction
page is a project called ``Labels and Descriptions'', which provides a
large list of Identifiers that are missing labels and descriptions in
different languages. Additional WikiProjects focus on specific topics,
such as LGBT-related content or items related to Ancient Greece, to
assist users with more specialized knowledge about these subjects to
find and develop data in those areas.

For users with advanced coding skills, Wikidata also welcomes the
contribution of bots to save community members time and effort. However,
since bots can make a lot of changes in a very short amount of time,
there are policies in place to limit the work they can do. All bots need
to have a limit on the number of changes made per minute, and their
edits need to be logged with a special bot flag, which can only be
obtained from a bureaucrat user and the approval from the Wikidata
community (``Wikidata:Bots'' 2019).

All the methods of contribution discussed so far is mostly for
individuals, but Wikidata also allows organizations to donate their
large quantities of data to the database. To make a ``Data Donation'',
the organization must reach out to the Wikidata community to determine
which data to add, and they must work with the community to import the
chosen data (``Wikidata:Data Donation'' 2019). It is interesting to note
that most organizations listed on Wikidata's ``Data Donation'' page
manage cultural heritage collections, such as UNESCO, the British
Museum, and the National Libraries of France and Australia. This will be
discussed further below, when I compare Wikidata with its sibling site,
Wikipedia.

\hypertarget{decentralization-of-conception-and-execution}{%
\section{Decentralization of Conception and
Execution}\label{decentralization-of-conception-and-execution}}

To discuss ideas and actions, the Wikidata Community has several
options. The main project chat is hosted on Wikidata as an individual
page, with a layout similar to the layout of a Wikipedia article. A user
can post a new ``discussion'' by adding a section to the page, and
responses are indented under the first post. After 7 days, all
discussions are added to the Archive Index, which acts as an open
repository for past problem solving and discussions (``Wikidata:Project
Chat'', 2019).

In addition to the main project chat, Wikidata provides similar pages
for special topics. Each governance role has its own ``Noticeboard'',
which allow other users to post a message for them to address and
respond to. Additional pages for requests of queries, deletions,
comments, bots, and permissions are also listed and routinely checked by
users responsible for those areas. When certain actions need community
approval, they are posted for discussion and voted for on the relevant
page.

By hosting these ``chats'' on Wikidata, the site allows for full
transparency of these discussions. Once a new post is added, all members
of the Wikidata community can see it without needing to log in or verify
credentials. Anyone can post a discussion, and anyone can respond to the
problem. However, the users who can \emph{fix} the problem may be
limited to roles with the correct permissions, such as deleting
problematic pages is limited to Administrators and changing permissions
is limited to Bureaucrats.

I will also note here that Wikidata does have an option for direct
communication. The community has a live chat, hosted on two different
websites: Telegram and IRC. Unfortunately, the author had technical
issues while trying to access these platforms, so the nature of these
chat forums are unknown.

\hypertarget{separation-of-governance-from-property-and-contract}{%
\section{Separation of Governance from Property and
Contract}\label{separation-of-governance-from-property-and-contract}}

According to Benkler (2016, pg. 2), this characteristic contains two
important parts, each of which will be addressed below:

\begin{itemize}
\item
  Inputs and outputs are open commons or common property
\item
  Governance utilizes combinations of participatory, meritocratic, and
  charismatic systems rather than proprietary or contractual models
\end{itemize}

\hypertarget{open-commons}{%
\subsection{Open Commons}\label{open-commons}}

This section can be addressed by looking at the copyright policies
associated with Wikidata and the larger Wikimedia Foundation. According
to ``Wikidata:Introduction'' (2018), all data in Wikidata is published
under the \emph{Creative Commons Public Domain Dedication 1.0}, so
content on Wikidata can be used, copied, and distributed without paying
fees or requesting permission. WikiMedia Terms and Conditions also
require that ``all submitted content be licensed so that it is freely
reusable by anyone who cares to access it'' (``Terms of Use'' 2019). By
establishing an open policy for submitted and presented content,
Wikidata and the larger WikiMedia Foundation fulfill the first bullet
listed above in ensuring that inputs and outputs are open commons.

\hypertarget{governance}{%
\subsection{Governance}\label{governance}}

To manage the community and the project, Wikidata has established five
governance roles: administrators, bureaucrats, checkusers, oversight,
and property creators. Each role handles a different aspect of managing
the Wikidata community, from deleting unwanted pages to adding and
taking away permissions from users. This creates a decentralized
leadership where the power is divided among different groups, rather
than held in one area. The different roles are as follows:

\begin{itemize}
\item
  \textbf{Administrators:} oversee deleting pages, blocking users,
  monitoring for vandalism, monitoring for role abuse, and returning
  vandalized pages to their pre-vandalized forms; requires a week of
  discussion and a community vote with at least eight supporting votes
  and over 75\% of total votes in support (``Wikidata:Administrators''
  2019).
\item
  \textbf{Bureaucrats:} can change permissions and rights for other
  users; requires a permissions request and at least 15 supporting votes
  in a community vote (with 80\% of total votes in favor) after a week
  of discussion (``Wikidata:Bureaucrats, 2017).
\item
  \textbf{CheckUsers:} can examine IP address information of users on
  Wikidata and block users in case of vandalism or ``sock puppetry'',
  where one user creates multiple accounts for improper use; requires a
  permissions request and at least 25 supporting votes (with 80\%
  support) after two weeks of discussion (``Wikidata:CheckUser'', 2019).
\item
  \textbf{Oversight:} also called ``suppression''; can hide and restore
  revisions, usernames in histories, and portions of log entries in a
  page's editing history in cases of hiding non-public personal
  information, hiding potentially libelous information, removing
  copyright violations, and removing blatant attack names in editing
  histories; requires a permissions request and at least 25 support
  votes (with 80\% support) after two weeks of discussion; rights is
  given by a bureaucrat formally requesting access from the Wikimedia
  Foundation (``Wikidata:Overwsight'' 2019).
\item
  \textbf{Property Creators:} can create properties; requires a
  nomination from themselves or someone else; user needs to be a trusted
  member of the community and have shown a satisfactory understanding of
  how Wikidata works; an administrator can add the flag
  (``Wikidata:Property Creators'', 2019).
\end{itemize}

As listed above, four of these roles require a period of discussion and
a voting process by the community members. The only exception is the
Property Creators, whose role was created when the Wikidata community
that decided only a select number of people should have technical
ability to create Properties (apart from the administrators), but the
new properties still require community consensus. By publicly posting
their interest, discussing their merits, and voting on new members of
each role, the Wikidata community fulfills Benkler's ``participatory,
meritocratic, and charismatic'' requirements mentioned earlier (Benkler
2016, pg. 2). However, Wikidata policies also limit the power of several
roles. For example, the Bureaucrats policy explicitly state that their
ability to grant permissions ``should not be used for political control;
to apply pressure on editors; or as a threat against another editor in a
dispute. There must be a valid reason to check a user''
(``Wikidata:CheckUser'', 2019). By dispersing the power and voting with
the consensus of the community on the user's merit, the governance is
very similar to a Meritocracy model (Gardler and Hanganu, 2013).

\hypertarget{wikidata-vs-wikipedia}{%
\section{Wikidata vs Wikipedia}\label{wikidata-vs-wikipedia}}

Wikidata has a strong relationship with Wikipedia. According to its
Introduction page (2018), Wikidata supports a structured, multilingual
data that can easily be reused by other Wikimedia projects. In a casual
discussion about the subject, Liaison Librarian Gina Bastone, who
organizes the Wikipedia Edit-a-thons hosted at the Perry-Castañeda
Library, explained to me that in her experience, Wikipedia is often used
to fill the content of Wikidata.

Such a close relationship could have an impact on the data bias within
Wikidata. Already, there have been several studies done on the
male-dominated demographics of Wikipedia (Glott, Schmidt, and Ghosh,
2010; Antin, Yee, Cheshire, and Nov, 2011). Another study by Konieczny
and Klein (2018), shows that these demographics is one of the major
factors involved in the lack of women and non-binary representation in
Wikipedia biographies. Even if the gender bias for Wikidata is unknown
(or at least, the author was unable to locate relevant data), it is
still incredibly possible that the gender bias of Wikipedia is impacting
the content of Wikidata.

The largest difference between Wikidata and Wikipedia is the structure
of information, which in turn shapes the communities interested
contributing to these databases. Wikidata allows a highly structured and
efficient method of presenting data, which is making it more popular
among academics studying information science, Natural Language
Processing, and medical industries (Mora-Cantallops, Sánchez-Alonso, \&
García-Barriocanal, 2019). Wikidata's structure has also caught the
attention of Galleries, Libraries, Archives, and Museums (GLAMs).

Earlier, I mentioned that many the data organizations listed on
Wikidata's ``Data Donations'' page were cultural heritage GLAMs. Since
2010, GLAMs have been involved in contributing to WikiProjects, with
Wikidata recently becoming the favored database (Stinson, Fauconnier,
and Wyatt, 2018). For this paper, I interviewed with Paloma Graciani
Picardo, the Metadata Librarian at the Harry Ransom Center, to discuss
the GLAMs' interest in Wikidata. According to her, ``Libraries and
archives see an opportunity to put their collections out there. They can
use the `archived at' {[}element{]} or, whenever you put something into
Wikidata, you can add a source''. She also described that some
librarians see Wikidata as a potential central hub of identifiers, where
catalog records could point to for additional information, other
relevant collections, and identifiers assigned by other organizations,
such as the Library of Congress or WorldCat. A similar project had been
started at the Harry Ransom Center, where she and a graduate student
were reconciling local names in the HRC Metadata Taxonomies with data
from Wikidata. However, there are still ongoing discussions about
whether all the data in a collection is appropriate for Wikidata, or if
Wikidata should focus in on major, widely-known concepts (Graciani
Picardo, 2019; Stinson, Fauconnier, and Wyatt, 2018). Although Wikidata
has a higher concentration of cultural heritage involvement, the
involvement usually comes from wealthy GLAMs with highly educated staff
from Europe and North America, so this continues to ``perpetuating the
dominance of colonialism-tainted collections and professional practices
in retelling history'', as phrased by Stinson, Fauconnier, and Wyatt
(2018). Cultural heritage institutions in Ghana, Indonesia, and
Argentine are working to upload their collections, which would help
balance out the bias, but there are still many challenges that need to
be addressed (Stinson, Fauconnier, and Wyatt, 2018).

\hypertarget{conclusion}{%
\section{Conclusion}\label{conclusion}}

Wikidata relies heavily upon the successful implementation of peer
production policies in order to create a large database of structured
data. It harnesses diverse motivations by allowing contributions to be
made easily by anyone. It decentralizes conception and execution of
ideas by allowing members multiple and specialized areas for topics to
be addressed, discussed, and voted upon. Finally, it also separates
government from property by making all data and discussions be open to
the public, and allowing transparent, open voting in almost all changes
within the community. These policies have allowed Wikidata to grow into
an more efficient database for supporting other Wikimedia sites and
GLAMs.

\hypertarget{references}{%
\section{References}\label{references}}

Antin, J., Yee, R., Cheshire, C., \& Nov, O. (2011). Gender Differences
in Wikipedia Editing. In \emph{Proceedings of the 7th International
Symposium on Wikis and Open Collaboration} (pp.~11--14). New York, NY,
USA: ACM.

Bastone, G. (2019). Personal Communication.

Benkler, Y. (2016) ``Peer Production and Cooperation'' in J. M. Bauer \&
M. Latzer (eds.), \emph{Handbook on the Economics of the Internet}.
Northampton, MA: Edward Elgar Publishing, Inc.

Gardler, R. and Hanganu, G. (2013). ``Governance Models'' on
\emph{OSSWatch}. Retrieved from
\url{http://oss-watch.ac.uk/resources/governancemodels}

Glott, R., Schmidt, P., \& Ghosh, R. (2010). Wikipedia survey--overview
of results. \emph{United Nations University: Collaborative Creativity
Group}, 1158-1178. Retrieved from
\url{http://www.ris.org/uploadi/editor/1305050082Wikipedia_Overview_15March2010-FINAL.pdf}

Graciani Picardo, P. (2019). Personal Interview.

Konieczny, P., \& Klein, M. (2018). Gender gap through time and space: A
journey through Wikipedia biographies via the Wikidata Human Gender
Indicator. \emph{New Media \& Society}, 20(12), 4608--4633.
\url{https://doi.org/10.1177/1461444818779080}

Mora-Cantallops, M., Sánchez-Alonso, S., \& García-Barriocanal, E.
(2019). ``A systematic literature review on wikidata''. \emph{Data
Technologies and Applications}, 53(3), 250-268.
\url{doi:10.1108/DTA-12-2018-0110}

Stinson, A. D., Fauconnier, S., \& Wyatt, L. (2018). Stepping beyond
libraries: The changing orientation in global GLAM-wiki. \emph{JLIS.it,
Italian Journal of Library and Information Science}, 9(3), 16-34.
\url{doi:10.4403/jlis.it-12480}

``Terms of Use'' (2019) on \emph{WikiMedia} Foundation. Retrieved from
\url{https://foundation.wikimedia.org/wiki/Terms_of_Use/en\#3._Content_We_Host}

``Wikidata'' (2019) on \emph{Wikidata}. Retrieved from
\url{https://www.wikidata.org/wiki/Q2013}

``Wikidata:Administrators'' (2019). On \emph{Wikidata}. Retrieved from
\url{https://www.wikidata.org/wiki/Wikidata:Administrators}

``Wikidata:Bots'' (2019). On \emph{Wikidata}. Retrieved from
\url{https://www.wikidata.org/wiki/Wikidata:Bots}

``Wikidata:Burcraucrats'' (2017). On \emph{Wikidata}. Retrieved from
\url{https://www.wikidata.org/wiki/Wikidata:Bureaucrats}

``Wikidata:CheckUser'' (2019) on \emph{Wikidata}. Retrieved from
\url{https://www.wikidata.org/wiki/Wikidata:CheckUser}

``Wikidata:Data Donation'' (2019) on \emph{Wikidata}. Retrieved from
\url{https://www.wikidata.org/wiki/Wikidata:Data_donation}

``Wikidata:Introduction'' (2018) on \emph{Wikidata}. Retrieved from
\url{https://www.wikidata.org/wiki/Wikidata:Introduction}

``Wikidata:Oversight'' (2019). On \emph{Wikidata}. Retrieved from
\url{https://www.wikidata.org/wiki/Wikidata:Oversight}

``Wikidata:Project Chat'' (2019) on \emph{Wikidata}. Retrieved from
\url{https://www.wikidata.org/wiki/Wikidata:Project_chat}

``Wikidata:Property Creators'' (2019). On \emph{Wikidata}. Retrieved
from \url{https://www.wikidata.org/wiki/Wikidata:Property_creators}


\end{document}
